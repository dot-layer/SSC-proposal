%%%%%%%%%%%%%%%%%%%%%%%%%%%%%%%%%%%%%%%%%
% Professional Formal Letter
% LaTeX Template
% Version 2.0 (12/2/17)
%
% This template originates from:
% http://www.LaTeXTemplates.com
%
% Authors:
% Brian Moses
% Vel (vel@LaTeXTemplates.com)
%
% License:
% CC BY-NC-SA 3.0 (http://creativecommons.org/licenses/by-nc-sa/3.0/)
%
%%%%%%%%%%%%%%%%%%%%%%%%%%%%%%%%%%%%%%%%%


%----------------------------------------------------------------------------------------
%	PACKAGES AND OTHER DOCUMENT CONFIGURATIONS
%----------------------------------------------------------------------------------------

\documentclass[11pt, a4paper]{letter} % Set the font size (10pt, 11pt and 12pt) and paper size (letterpaper, a4paper, etc)

%\graphicspath{{./fig/}} %fig path NOT WORKING


\input{structure.tex} % Include the file that specifies the document structure

\longindentation=0pt % Un-commenting this line will push the closing "Sincerely," and date to the left of the page

%----------------------------------------------------------------------------------------
%	YOUR INFORMATION
%----------------------------------------------------------------------------------------

%\Who{} % Your name

%\Title{} % Your title, leave blank for no title

\authordetails{%
%	Département de mathematiques\\ et de statistique\\ % Your department/institution
	Samuel Perreault\\
	623 rue Du Moulin\\ % Your address
	Mont-Tremblant, Qc, J8E 2V1\\ % Your city, zip code, country, etc
	Email:  communications@dotlayer.org\\ % Your email address
	Phone: (819) 425-0303\\ % Your phone number
	URL: dotlayer.org % Your URL
}

%----------------------------------------------------------------------------------------
%	HEADER CONTENTS
%----------------------------------------------------------------------------------------

\logo{logo.png} % Logo filename, your logo should have square dimensions (i.e. roughly the same width and height), if it does not, you will need to adjust spacing within the HEADER STRUCTURE block in structure.tex (read the comments carefully!)

\headerlineone{} % Top header line, leave blank if you only want the bottom line

\headerlinetwo{.Layer} % Bottom header line

%----------------------------------------------------------------------------------------

\begin{document}

%----------------------------------------------------------------------------------------
%	TO ADDRESS
%----------------------------------------------------------------------------------------

\begin{letter}{
	Prof. Mary Thompson\\
	Chair of the Fundraising Committee\\
	Statistical Society of Canada
	
	\bigskip
	\textbf{Subject: education and outreach proposal}%
}

%----------------------------------------------------------------------------------------
%	LETTER CONTENT
%----------------------------------------------------------------------------------------


\opening{Dear Prof. Thompson,}

%\emph{CALL: This is a new call for proposals for advancing the goals of the SSC to promote statistics and probability to students and educators, or to help students and young researchers develop careers in the statistical sciences.}
%
%\emph{The SSC Board will award between 500 and 5,000 for the best proposals received by January 31, 2019. Preference will be given to proposals that have a broad reach, e.g. are regional or national rather than local, and that are able to obtain additional support. The number of awards will depend on the funds available. If a proposal involves electronic services, proposers are encouraged to contact the Electronic Services Manager of the SSC for advice.}

I am writing to you on behalf of \textit{.Layer} (pronounced ``dot layer''), a community created for promoting collaboration and knowledge sharing in the field of data science. As an organization in its creation process, we were very pleased to learn about the recent call for education and outreach proposals of the Statistical Society of Canada. I will try, in this letter, to clearly describe who we are, what our mission is, and how we intend to put our words into action. We hope that you will find our proposal convincing enough to grant us your support.

%\pagebreak
%
%\emph{QUESTION: Your idea and a rationale, explaining how it fits with the mission of the SSC:}
%   
%\emph{INFO: The Statement of Purpose of the SSC is to advance the knowledge and education of Canadians concerning statistical sciences and related fields.}

%As a statistics enthusiast, I rarely doubt my decision of pursuing graduate studies in statistics. However, it is hard to deny, as you may have experienced yourself, that such enterprise sometimes comes with its lot of solitude. To my great surprise,

\bigskip
\noindent \textbf{Who we are}

The project started during the 2018 Summer, when some of us decided to organize a week-end of presentations and activities related to data science. The high-level motivation of this meeting was to bring together friends of friends for a brainstorm about how to unite the different data science communities we were part of. Fifteen people responded to the call and out of it emerged .Layer, an open community looking forward for anyone to join in and contribute. It now consists of more or less **[??]** students and/or practitioners in the fields of statistics, actuarial science and computer science, residing in either Montreal or Quebec City. Members responsible for the present proposal are:
\begin{enumerate}
	\item[] \textbf{David Beauchemin, B.Sc.}\\
	\quad Masters student at Laval University (computer science)
	%\\\quad Member of university reseach labs: GRAAL
	\item[] \textbf{Christopher Blier-Wong, M.Sc.}\\
	\quad PhD student at Laval University (actuarial science)\\
	\quad Masters student at Laval University (computer science)
	%\\\quad Member of university reseach labs: ACT\&RISK, QUANTACT and BDRC
	\item[] \textbf{Antoine Buteau}\\
	\quad **[??]**
	\item[] \textbf{Stéphane Caron}\\
	\quad Actuarial analyst at Intact Insurance\\
	\quad Masters student at Laval University (mathematics and statistics)
	\item[] \textbf{Marie-Pier Côté}\\
	\quad **[??]**
	\item[] \textbf{Nicolas Garneau}\\
	\quad **[??]**
	\item[] \textbf{Samuel Perreault}\\
	\quad PhD student at Laval University (mathematics and statistics)\\
	\quad Research \& innovation analyst at The Cooperators Insurance
\end{enumerate}

\noindent All of us affiliated to Laval University's mathematics and statistics, and actuarial science departments are members of the SSC, with Marie-Pier Côté being part of the committee on new investigators as well. Three of us, Marie-Pier Côté, Christopher Blier-Wong and myself (Samuel Perreault) presented at the 2018 SSC meeting.


We are currently in the process of registering a nonprofit organization for administering the community's activities. Here is a sketch of the structure we intend to put in place.

\begin{enumerate}
	\item[] \textbf{Board of directors}: The board consists of four people: an executive president, a secretary, a treasurer and a vice-president assigned to communication.
	\item[] \textbf{Members}: In addition to contributing and participating to the organization's projects and events, members have the right to vote during the general assemblies. One must first be accepted by the current members in order to gain its member status.
	\item[] \textbf{Supporters**[Sympathizers?]**}: This group includes any other individual involved in the community's projects and events. They can share their opinion, but have no right to vote during assemblies.
\end{enumerate}

This simple structure allows the organization to be as inclusive as possible while assuring that funds and efforts are spent on activities that are aligned with its mission. Any supporter minimally involved in the community, and demonstrating that he or she shares its inclusive values, should expect to be accepted as a member if requested.

\bigskip
\noindent \textbf{Our mission}

The main objective behind .Layer is to create both virtual and physical meeting places, for the curious as well as the experimented, to learn, share and exchange ideas related to data science; hence the slogan \emph{meet, learn, \& share}. We did not identify a sole method for achieving this objective, but committed to help each other in our current and future projects that are aligned with it. So far, we have been working mostly on the two following initiatives.
\begin{enumerate}
	\item[(a)] \emph{Meetup Machine Learning Québec}: This project was founded by Nicolas Garneau. As the name suggests, it is dedicated to the organization of meetups and conferences related to machine learning in Quebec City. Before .Layer even existed, \emph{Meetup} had already organized three events in 2017-2018 (at Laval University/XpertSea's offices/Coveo's offices), attracting more than 100 people in total. Since the creation of .Layer, Nicolas joined forces with many others to organize events of even greater scope. Concretely, we organized both a conference (Laval University) and a hackathon (**[STARTUP]**'s offices) on statistics and machine learning applied to the insurance domain; as well as a workshop on natural language processing (Laval University). These events attracted an approximate total of **[??]**. They were organized on a voluntary basis and were paid for (rooms, food, beverages, web services for GPUs, etc.) by our sponsors: Intact Insurance, Co-operators Insurance, **[STARTUP]** and, from Laval University, the Big Data Research Center (CRDM), the Chaire d'actuariat and the service de placements (SPLA).
	\item[(b)] \emph{.Layer blog}: During the founding week-end of .Layer, we agreed that a blog would be great as a first project to collaborate on, as well as a convenient way of reporting our activities and sharing our ideas. We ourselves were surprised by how quick the project evolved: the blog is already up and running at dotlayer.org.
\end{enumerate}

We seek your support for helping us continue structure the community; improve our website; organize more educational events; and kickoff new projects. These four axes are detailed below.

%\emph{QUESTION: Who will implement the project and how it will be implemented}
\begin{enumerate}
	\item \textbf{Community organization}
	
	\quad We are lucky to have among us a professional graphic designer, Jean-Christophe Yelle, who created, at no cost, the logo that you can see in the head of the letter. Among the other things we wish to integrate to the community are personalized e-mail addresses (@dotlayer.org) for the administering members and +++++++ **[WHAT MORE? - @steph @bute - membership fees here?]**
	
	\bigskip
	\emph{Member in charge:} Stéphane Caron\\
	
	\item \textbf{Blog}
	
	\quad We are now working on a bilingual display of the website (en/fr). So far, some of the articles are in English, some in French. In order to be as inclusive as possible, we would like to have the important ones in both languages. We can and will do it ourselves, but a little review from a professional would be of great use.
	
	\quad As soon as the translation challenge will be behind us, we will focus on a strategy for promoting the new articles and the website in general. **[HOW will we spread the word?]**
	
	
	\bigskip
	\emph{Member in charge:} Antoine Buteau\\
	
	\item \textbf{Educational events}.
	
	\quad One of the members of .Layer, Nicolas Garneau, is the founder of \emph{Meetup	Machine Learning Québec}, 
	
\quad We are seeking new funds to organize similar events, this time related to sports analytics **[TBD]**.
	
%	The latter kind of events, by its nature, makes it a bit easier to find sponsors as it also is a good way for corporations to recruit new talents. A second kind of activities we wish to organize, and for which we think your funds would be of great help, focuses on a younger audience. Indeed, we would like to ** Event in schools all over the province?
**[MORE ON: WE WANT TO EDUCATE THE GENERAL PUBLIC ABOUT WHAT DATA SCIENCE CAN DO FOR THEM.]**
			
	\bigskip
	\emph{Member in charge:} Nicolas Garneau\\
	
	\item \textbf{New projects}.
	
	\quad Podcast. Put focus on the audience (young people interested by studies/careers in data science).
	
	\bigskip
	\emph{Member in charge:} David Beauchemin
\end{enumerate}

In addition to the funds obtained from sponsors, we intend to ask voluntary contributions from our members/sympathizers. Your support would hence help us reduce such burden on individuals.


\bigskip
\noindent \textbf{Administration of the funds}

Administering the funds and reporting how they are used is the \emph{raison d'\^{e}tre} of the nonprofit entity tied to .Layer. Until its official incorporation, I, Samuel Perreault, take the responsability of providing such report to the Statistical Society of Canada. With this letter are provided a budget and a timeline for the administration of the funds. **[PROVIDE the budget and timeline!]**

\bigskip

**[CONCLUSION/SUMMARY]**

%\emph{QUESTION: If the project involves website design, where the site will be hosted}
%
%This is answered right?
      
\closing{Sincerely,}

%----------------------------------------------------------------------------------------
%	OPTIONAL FOOTNOTE
%----------------------------------------------------------------------------------------

% Uncomment the 4 lines below to print a footnote with custom text
%\def\thefootnote{}
%\def\footnoterule{\hrule}
%\footnotetext{\hspace*{\fill}{\footnotesize\em Footnote text}}
%\def\thefootnote{\arabic{footnote}}

%----------------------------------------------------------------------------------------

\end{letter}

\end{document}
