%%%%%%%%%%%%%%%%%%%%%%%%%%%%%%%%%%%%%%%%%
% Professional Formal Letter
% LaTeX Template
% Version 2.0 (12/2/17)
%
% This template originates from:
% http://www.LaTeXTemplates.com
%
% Authors:
% Brian Moses
% Vel (vel@LaTeXTemplates.com)
%
% License:
% CC BY-NC-SA 3.0 (http://creativecommons.org/licenses/by-nc-sa/3.0/)
%
%%%%%%%%%%%%%%%%%%%%%%%%%%%%%%%%%%%%%%%%%


%----------------------------------------------------------------------------------------
%	PACKAGES AND OTHER DOCUMENT CONFIGURATIONS
%----------------------------------------------------------------------------------------

\documentclass[11pt, a4paper]{letter} % Set the font size (10pt, 11pt and 12pt) and paper size (letterpaper, a4paper, etc)


\input{structure.tex} % Include the file that specifies the document structure

\longindentation=0pt % Un-commenting this line will push the closing "Sincerely," and date to the left of the page

%----------------------------------------------------------------------------------------
%	YOUR INFORMATION
%----------------------------------------------------------------------------------------

\Who{Samuel Perreault} % Your name

\Title{} % Your title, leave blank for no title

\authordetails{
%	Département de mathematiques\\ et de statistique\\ % Your department/institution
	623 rue Du Moulin\\ % Your address
	Mont-Tremblant, Qc, J8E 2V1\\ % Your city, zip code, country, etc
	Email:  dotlayerorg@gmail.com\\ % Your email address
	Phone: (819) 425-0303\\ % Your phone number
	URL: dotlayer.org % Your URL
}

%----------------------------------------------------------------------------------------
%	HEADER CONTENTS
%----------------------------------------------------------------------------------------

\logo{logo.jpg} % Logo filename, your logo should have square dimensions (i.e. roughly the same width and height), if it does not, you will need to adjust spacing within the HEADER STRUCTURE block in structure.tex (read the comments carefully!)

\headerlineone{} % Top header line, leave blank if you only want the bottom line

\headerlinetwo{.Layer} % Bottom header line

%----------------------------------------------------------------------------------------

\begin{document}

%----------------------------------------------------------------------------------------
%	TO ADDRESS
%----------------------------------------------------------------------------------------

\begin{letter}{
	Prof. Mary Thompson\\
	Chair of the Fundraising Committee\\
	Statistical Society of Canada
	
	\bigskip
	\textbf{Subject: education and outreach proposal}%
}

%----------------------------------------------------------------------------------------
%	LETTER CONTENT
%----------------------------------------------------------------------------------------


\opening{Dear Prof. Thompson,}

%\emph{CALL: This is a new call for proposals for advancing the goals of the SSC to promote statistics and probability to students and educators, or to help students and young researchers develop careers in the statistical sciences.}
%
%\emph{The SSC Board will award between 500 and 5,000 for the best proposals received by January 31, 2019. Preference will be given to proposals that have a broad reach, e.g. are regional or national rather than local, and that are able to obtain additional support. The number of awards will depend on the funds available. If a proposal involves electronic services, proposers are encouraged to contact the Electronic Services Manager of the SSC for advice.}

I am writing you on behalf of .Layer (pronounced ``dot layer''), a community created for promoting collaboration and knowledge sharing in the field of data science. As an organization in its creation process, we were very pleased to learn about the recent call for education and outreach proposals of the Statistical Society of Canada. I will try, in this letter, to clearly describe who we are, what our mission is, and how we intend to put our words into action. We hope that you will find our proposal convincing enough to grant us your support.

%\pagebreak
%
%\emph{QUESTION: Your idea and a rationale, explaining how it fits with the mission of the SSC:}
%   
%\emph{INFO: The Statement of Purpose of the SSC is to advance the knowledge and education of Canadians concerning statistical sciences and related fields.}

%As a statistics enthusiast, I rarely doubt my decision of pursuing graduate studies in statistics. However, it is hard to deny, as you may have experienced yourself, that such enterprise sometimes comes with its lot of solitude. To my great surprise,

\bigskip
\noindent \textbf{Who we are}

We define ourselves as an open community, looking forward for anyone to join in and contribute. The community now consists of fifteen **[TBD]** students and/or practitioners in the fields of statistics, actuarial science and computer science, residing in either Montreal or Quebec City. The project started during the 2018 Summer, when some of us decided to organize a week-end of presentations and activities related to data science. The high-level motivation of this meeting was to bring together friends of friends for a brainstorm about how to unite the different data science communities we were part of. Out of it emerged .Layer.

We are currently in the process of registering a nonprofit organization and creating a board of directors for administering the community's activities. Here is a sketch of the structure we intend to put in place.

\begin{enumerate}
	\item \textbf{Board of directors}: The board consists of 4 people. There is a executive president, a secretary, a treasurer and a vice-president assigned to communication.
	\item \textbf{Members}: The organization is formed of members that are contributing and participating to the organization projects and events. The members must be accepted by other members and have right to vote to some decisions taken by the organization.
	\item \textbf{Supporters}: This inclusive group is formed of people sharing and promoting the mission of the community. They consist of anyone wanting to be part of the community project and events. They can share their opinion, but have no right to vote during organization assemblies.
\end{enumerate}

This structure makes the organization very inclusive to anybody wanting to contribute to this scientific community. At the same time, the administrative and monetary decisions are taken by a small group of people. That should contributes to align more closely the decisions to the mission of the group.

\bigskip
\noindent \textbf{Our mission}

The main objective behind .Layer is to create both virtual and physical meeting places, for the curious as well as the experimented, to learn, share and exchange ideas related to data science; hence the slogan \emph{meet, learn, \& share}. We did not identify a sole method for achieving this objective, but committed to help each other in our current and future projects that are aligned with it. So far, we have been working mostly on the two following initiatives.
\begin{enumerate}
	\item[(a)] \emph{Meetup Machine Learning Québec}: This project was founded by Nicolas Garneau before .Layer even existed. As the name suggests, it is dedicated to the organization of meetups and conferences related to machine learning. **[BRIEFLY write what was done before .Layer]** Since the creation of .Layer, we have built a strong team around Nicolas to help him organize events of greater scope. Concretely, we organized three events in the city of Quebec: both a conference and a hackathon on statistics and machine learning applied to the insurance domain where more than a 125 **[CHECK TOTAL ATTENDEES]** students and practitioners showed up; as well as a workshop on natural language processing for which we had to stop taking registrations after our 75 **[CHECK TOTAL ATTENDEES]** spots literally disappeared. All these initiatives were organized on a voluntary basis and were paid for (rooms, food, beverages, web services for GPUs, etc.) by sponsors: Intact Insurance, Co-operators Insurance, STARTUP and the Big Data Research Center (CRDM) of Laval University.
	\item[(b)] \emph{.Layer blog}: During the founding week-end of .Layer, we agreed that a blog would be great as a first project to collaborate on, as well as a convenient way of reporting our activities and sharing our ideas. We ourselves were surprised by how quick the project evolved as the blog is already up and running at dotlayer.org.
\end{enumerate}

We seek your support for helping us continue the organization the community; organize more educational events; and kickoff new projects. These three axes are detailed below.

%\emph{QUESTION: Who will implement the project and how it will be implemented}
\begin{enumerate}
	\item \textbf{Community organization}
	
	\quad We are lucky to have among us a professional graphic designer, Jean-Christophe Yelle, who created, at no cost, the logo that you can see in the head of the letter. Among the other things we wish to integrate to the community are personalized e-mail addresses (@dotlayer.org) for the administering members and +++++++ **[WHAT MORE? - @steph @bute - membership fees here?]**
	
	\bigskip
	\emph{Member in charge:} Stéphane Caron\\
	
	\item \textbf{Blog}
	
	\quad We are now working on a bilingual display of the website (en/fr). So far, some of the articles are in English, some in French. In order to be as inclusive as possible, we would like to have the important ones in both languages. We can and will do it ourselves, but a little review from a professional would be of great use.
	
	\quad As soon as the translation challenge will be behind us, we will focus on a strategy for promoting the new articles and the website in general. **[HOW will we spread the word?]**
	
	
	\bigskip
	\emph{Member in charge:} Antoine Buteau\\
	
	\item \textbf{Educational events}.
	
	\quad One of the members of .Layer, Nicolas Garneau, is the founder of \emph{Meetup	Machine Learning Québec}, 
	
\quad We are seeking new funds to organize similar events, this time related to sports analytics **[TBD]**.
	
%	The latter kind of events, by its nature, makes it a bit easier to find sponsors as it also is a good way for corporations to recruit new talents. A second kind of activities we wish to organize, and for which we think your funds would be of great help, focuses on a younger audience. Indeed, we would like to ** Event in schools all over the province?
**[MORE ON: WE WANT TO EDUCATE THE GENERAL PUBLIC ABOUT WHAT DATA SCIENCE CAN DO FOR THEM.]**
			
	\bigskip
	\emph{Member in charge:} Nicolas Garneau\\
	
	\item \textbf{New projects}.
	
	\quad Podcast. Put focus on the audience (young people interested by studies/careers in data science).
	
	\bigskip
	\emph{Member in charge:} David Beauchemin
\end{enumerate}

In addition to the funds obtained from sponsors, we intend to ask voluntary contributions from our members/sympathizers. Your support would hence help us reduce such burden on individuals.


\bigskip
\noindent \textbf{Administration of the funds}

Administering the funds and reporting how they are used is the \emph{raison d'\^{e}tre} of the nonprofit entity tied to .Layer. Until its official incorporation, I, Samuel Perreault, take the responsability of providing such report to the Statistical Society of Canada. With this letter are provided a budget and a timeline for the administration of the funds. **[PROVIDE the budget and timeline!]**

\bigskip

**[CONCLUSION/SUMMARY]**

%\emph{QUESTION: If the project involves website design, where the site will be hosted}
%
%This is answered right?
      
\closing{Sincerely,}

%----------------------------------------------------------------------------------------
%	OPTIONAL FOOTNOTE
%----------------------------------------------------------------------------------------

% Uncomment the 4 lines below to print a footnote with custom text
%\def\thefootnote{}
%\def\footnoterule{\hrule}
%\footnotetext{\hspace*{\fill}{\footnotesize\em Footnote text}}
%\def\thefootnote{\arabic{footnote}}

%----------------------------------------------------------------------------------------

\end{letter}

\end{document}
