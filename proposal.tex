%%%%%%%%%%%%%%%%%%%%%%%%%%%%%%%%%%%%%%%%%
% Professional Formal Letter
% LaTeX Template
% Version 2.0 (12/2/17)
%
% This template originates from:
% http://www.LaTeXTemplates.com
%
% Authors:
% Brian Moses
% Vel (vel@LaTeXTemplates.com)
%
% License:
% CC BY-NC-SA 3.0 (http://creativecommons.org/licenses/by-nc-sa/3.0/)
%
%%%%%%%%%%%%%%%%%%%%%%%%%%%%%%%%%%%%%%%%%


%----------------------------------------------------------------------------------------
%	PACKAGES AND OTHER DOCUMENT CONFIGURATIONS
%----------------------------------------------------------------------------------------

\documentclass[11pt, a4paper]{letter} % Set the font size (10pt, 11pt and 12pt) and paper size (letterpaper, a4paper, etc)

%\graphicspath{{./fig/}} %fig path NOT WORKING
\usepackage{hyperref}


\input{structure.tex} % Include the file that specifies the document structure

\longindentation=0pt % Un-commenting this line will push the closing "Sincerely," and date to the left of the page

%----------------------------------------------------------------------------------------
%	YOUR INFORMATION
%----------------------------------------------------------------------------------------

%\Who{} % Your name
%\Title{} % Your title, leave blank for no title

\authordetails{%
%	Département de mathematiques\\ et de statistique\\ % Your department/institution
	Samuel Perreault\\
	623 rue Du Moulin\\ % Your address
	Mont-Tremblant, Qc, J8E 2V1\\ % Your city, zip code, country, etc
	Email:  communications@dotlayer.org\\ % Your email address
	Phone: (819) 425-0303\\ % Your phone number
	URL: dotlayer.org % Your URL
}

%----------------------------------------------------------------------------------------
%	HEADER CONTENTS
%----------------------------------------------------------------------------------------

\logo{logo.png} % Logo filename, your logo should have square dimensions (i.e. roughly the same width and height), if it does not, you will need to adjust spacing within the HEADER STRUCTURE block in structure.tex (read the comments carefully!)

\headerlineone{} % Top header line, leave blank if you only want the bottom line

\headerlinetwo{.Layer} % Bottom header line

%----------------------------------------------------------------------------------------

\begin{document}

%----------------------------------------------------------------------------------------
%	TO ADDRESS
%----------------------------------------------------------------------------------------

\begin{letter}{
	Prof. Mary Thompson\\
	Chair of the Fundraising Committee\\
	Statistical Society of Canada
	
	\bigskip
	\textbf{Subject: education and outreach proposal}%
}

%----------------------------------------------------------------------------------------
%	LETTER CONTENT
%----------------------------------------------------------------------------------------


\opening{Dear Prof. Thompson,}

%\emph{CALL: This is a new call for proposals for advancing the goals of the SSC to promote statistics and probability to students and educators, or to help students and young researchers develop careers in the statistical sciences.}
%
%\emph{The SSC Board will award between 500 and 5,000 for the best proposals received by January 31, 2019. Preference will be given to proposals that have a broad reach, e.g. are regional or national rather than local, and that are able to obtain additional support. The number of awards will depend on the funds available. If a proposal involves electronic services, proposers are encouraged to contact the Electronic Services Manager of the SSC for advice.}

I am writing to you on behalf of \textit{.Layer} (pronounced ``dot layer''), a community created for promoting collaboration and knowledge sharing in the field of data science. As an organization in its creation process, we were very pleased to learn about the recent call for education and outreach proposals of the Statistical Society of Canada. I will try, in this letter, to clearly describe who we are, what our mission is, and how we intend to put our words into action. We hope that you will find our proposal convincing enough to grant us your support.

%\pagebreak
%
%\emph{QUESTION: Your idea and a rationale, explaining how it fits with the mission of the SSC:}
%   
%\emph{INFO: The Statement of Purpose of the SSC is to advance the knowledge and education of Canadians concerning statistical sciences and related fields.}

%As a statistics enthusiast, I rarely doubt my decision of pursuing graduate studies in statistics. However, it is hard to deny, as you may have experienced yourself, that such enterprise sometimes comes with its lot of solitude. To my great surprise,

\bigskip
\noindent \textbf{Who we are}

Last Summer, some of us organized a week-end of presentations and activities related to data science. Friends of friends, fifteen in total, then met for an epic brainstorm. %Our ultimate goal was to design a framework allowing us to share ideas and collaborate on projects more easily, both between ourselves and with any data science enthusiast out there. And so \emph{.Layer} was created. 
Out of it emerged \emph{.Layer}, a Quebec City hub consisting of more or less 25 students and/or practitioners in the fields of statistics, actuarial science and computer science, (mostly) residing in either Montreal or Quebec City. We create, through various initiatives, both virtual and physical spaces where people can meet, learn and share. Members responsible for the present proposal are (* = SSC member):
\begin{enumerate}
	\item[] \textbf{Jean-Thomas Baillargeon, M.Sc. FSA, CERA}\\
	\quad PhD student at Laval University (computer science)\\
	\quad Lecturer at Laval University\\
	\quad Former actuarial analyst at Industrial Alliance (6 years)\\
	\quad Former data product manager at Xpertsea Solution (2 years)\\
	\item[] \textbf{David Beauchemin, B.Sc.}\\
	\quad Masters student at Laval University (computer science)
	\item[] \textbf{Christopher Blier-Wong*, M.Sc.}\\
	\quad PhD student at Laval University (actuarial science)\\
	\quad Masters student at Laval University (computer science)
	%\\\quad Member of university reseach labs: ACT\&RISK, QUANTACT and BDRC
	\item[] \textbf{Antoine Buteau}\\
	\quad **[??]**
	\item[] \textbf{Stéphane Caron*, B.Sc.}\\
	\quad Actuarial analyst at Intact Insurance\\
	\quad Masters student at Laval University (mathematics and statistics)
	\item[] \textbf{Marie-Pier Côté*, PhD, FSA}\\
	\quad Assistant professor, École d'actuariat, Laval University
	\quad Teaching chair on big data analysis for actuariat science --- Intact
	\quad Member of the \emph{New Investigators Committee} of the \emph{SSC}
	\item[] \textbf{Nicolas Garneau}\\
	\quad **[??]**
	\item[] \textbf{Samuel Perreault*, M.Sc.}\\
	\quad PhD student at Laval University (mathematics and statistics)\\
	\quad Research \& innovation analyst at The Cooperators Insurance
\end{enumerate}

\noindent Three of us, Marie-Pier Côté, Christopher Blier-Wong and Samuel Perreault presented at the 2018 SSC meeting.


We are currently in the process of registering a nonprofit organization for administering the community's activities. Here is a sketch of the structure we intend to put in place.

\begin{enumerate}
	\item[] \textbf{Board of directors}: The board consists of five people: an executive president, a secretary, a treasurer, a vice-president assigned to communication and a vice-president assigned to external affairs.
	\item[] \textbf{Members}: In addition to contributing and participating to the organization's projects and events, members have the right to vote during the general assemblies. One must first be accepted by the current members in order to gain its member status.
	\item[] \textbf{Sympathizers}: This group includes any other individual involved in the community's projects and events. They can share their opinion, but have no right to vote during assemblies.
\end{enumerate}

This simple structure allows the organization to be as inclusive as possible while assuring that funds and efforts are spent on activities that are aligned with its mission. Any sympathizer minimally involved in the community, and demonstrating that he or she shares its inclusive values, should expect to be accepted as a member if requested.

\bigskip
\noindent \textbf{Our mission}

As stated, the main objective behind \emph{.Layer} is to create both virtual and physical spaces for the curious as well as the experimented. So far, we have been working mostly on the two following initiatives.
\begin{enumerate}
	\item[(a)] \emph{Meetup Machine Learning Québec}: This project was founded by Nicolas Garneau. As the name suggests, it is dedicated to the organization of meetups and conferences related to machine learning in Quebec City. Before \emph{.Layer} even existed, \emph{Meetup} had already organized three events in 2017-2018 (at Laval University/XpertSea's offices/Coveo's offices), attracting more than 100 people in total. Since then, Nicolas joined forces with many others to organize events of even greater scope. Concretely, we organized both a conference (Laval University) and a hackathon (Bloc Solutions' offices) on statistics and machine learning applied to casualty insurance; as well as a workshop on natural language processing (Laval University). These events attracted an approximate total of 200 people. They were paid for (rooms, food, beverages, web services for GPUs, etc.) by our sponsors: Intact Insurance, Co-operators Insurance, Bloc Solutions and, from Laval University, the Big Data Research Center (CRDM), the Chaire d'actuariat and the service de placements (SPLA).
	\item[(b)] \emph{.Layer blog}: During the founding week-end of \emph{.Layer}, we agreed that a blog would be great as a first project to collaborate on, as well as a convenient way of reporting our activities and sharing our ideas. Though great devotion: the blog is already up and running at \href{https://www.dotlayer.org/}{dotlayer.org}. In order to reach as many people Canada-wide, we have recently made the website bilingual (en/fr).
\end{enumerate}

We seek your support for helping on the four matters detailed below.

\begin{enumerate}
	\item \textbf{Structuring the community}
	
	\quad We are lucky to have among us a professional graphic designer, Jean-Christophe Yelle, who created our logo at no cost. However, we believe that professional services not directly related to data science (like logo design, t-shirt design, social media setup, and so on) should be fairly remunerated. In the budget provided with the letter, we also detail recent and future expenses related to the registration of \emph{.Layer} as a nonprofit organization.
	
	\bigskip
	\emph{Member in charge:} Stéphane Caron\\
	
	\item \textbf{.Layer blog}
	
	\quad We are now working on a bilingual display of the website (en/fr). So far, some of the articles are in English, some in French. In order to be as inclusive as possible, we would like to have the important ones in both languages. A revision of the website by a professional would be of great use. ****
	
	\bigskip
	\emph{Member in charge:} Antoine Buteau\\
	
	\item \textbf{Educational events}.
	
	\quad More \emph{Meetup Machine Learning Québec} events are coming. An introduction to the foundations machine learning, followed by a workshop on computer vision are offered on February 7 and 9 respectively. We are always seeking funds for the organization of more events of the kind. More details are provided in the budget. keep price low.****
			
	\bigskip
	\emph{Member in charge:} Nicolas Garneau\\
	
	\item \textbf{New projects}.
	
	\quad Some of us are already working on a francophone podcast named \emph{OpenLayer}. In video clips of about 45-60 minutes, David Beauchemin would meet different actors of the data science community to discuss the journey that took them where they are today and what their day-to-day looks like. The objective is to shed light on the job market, a hidden layer for aspiring data scientists. David could for example meet a Ph.D. student in statistics or artificial intelligence or a data analyst at some insurance company. The concept is influenced by podcasts \emph{à la} Joe Rogan and ``Mike Ward sous écoute''. In each episode, we will discover someone and his/her surroundings through a casual conversation.
	
	\bigskip
	\emph{Member in charge:} David Beauchemin
\end{enumerate}

In addition to the funds obtained from sponsors, we intend to ask voluntary contributions from our members/sympathizers. Your support would hence help us reduce such burden on individuals.


\bigskip
\noindent \textbf{Administration of the funds}

Administering the funds and reporting how they are used is the \emph{raison d'\^{e}tre} of the nonprofit entity tied to .Layer. Until its official incorporation, the members responsible for the proposal will take the responsability of providing such report to the \emph{Statistical Society of Canada}. With this letter is provided a budget for the coming months, which includes a timeline.

\bigskip

We strongly believe that the goals of \emph{.Layer} are aligned with the mission of the \emph{SSC}. Through our workshops of varying levels and our educational blog posts, we help to broadcast the importance of sound statistical methods **[and workflow?]** for data analysis. The blog acts as a platform for sharing ideas and reinforcing the sense of community between all individuals interested in data science and related fields. The meetup events and the workshops are also great opportunities to foster exchanges between theoreticians and practitioners.

On behalf of everyone involved in \emph{.Layer}, I would like to thank you for considering our proposal. We wish to have convinced you that our projects are worthy of your support.

\closing{Sincerely,}

\noindent Samuel Perreault

%----------------------------------------------------------------------------------------
%	OPTIONAL FOOTNOTE
%----------------------------------------------------------------------------------------

% Uncomment the 4 lines below to print a footnote with custom text
%\def\thefootnote{}
%\def\footnoterule{\hrule}
%\footnotetext{\hspace*{\fill}{\footnotesize\em Footnote text}}
%\def\thefootnote{\arabic{footnote}}

%----------------------------------------------------------------------------------------

\end{letter}

\end{document}
